\documentclass{ximera}

\usepackage{amsmath}	
\usepackage{imakeidx}
\usepackage{amsthm}
\usepackage{hyperref}
\usepackage{pgfplots}
\usepackage{tikz}
\usepackage{float}
\usepackage{subcaption}
\usepackage[margin=95pt]{geometry}


\newtheorem{ass}{Assumption}


\begin{document}

\chapter{Introduction}

\section{Futures}

This section deals with derivatives that are called {\bf futures contracts}. Futures have a lot in common with forward contracts; however, there are some distinct differences as well. Some similarities that futures have with forward contracts are that they both have
\begin{enumerate}
\item an underlying asset,
\item a time to expiration,
\item a buyer and a seller,
\item and an agreed upon price.
\end{enumerate} 
The agreed upon price is called the {\bf futures price} in a futures contract. It is the spot price of the futures contract.

Some of the differences between futures and forward contracts are that futures are
\begin{enumerate}
\item regulated by the market (forward contracts are tailored to the buyer and seller),
\item traded on exchanges (forward contracts don't have exchanges),
\item and marked to market.
\end{enumerate}
One of the regulations that markets require of futures exchanges is that buyers must make a margin deposit for to enter into a position. This is to protect the sellers in the exchange. {\bf Marking to market} is the process where gains and losses to the futures position are credited (debited) to the margin account. The credits and debits are absolute; they are not some proportion of the total value of the contract. Both the buyer and seller must open a margin account to participate in the exchange. Typically, the margin required is small relative to the size of the contract.

A margin account must maintain a certain value to keep all participants in the exchange. This value is called {\bf maintenance margin}. The maintenance margin will typically be given in terms of a percentage of the initial margin account value; however, it is possible to be given the value as a dollar amount. In the event that one of the parties' margin account falls in value below the mintenance margin, a {\bf margin call} is initiated. A margin call requires the party that received the call to deposit an amount of collatera so that the account is back to its initial value. 

A lot has been said, but little has been done in terms of numerics. Let's try an example.

\begin{example}
You would like to invest in the S\&P 500 index. To do so, you can enter into a futures contract on the same exchange. A typical S\&P 500 futures contract costs 250 times the value of the S\&P 500 index. The initial margin is 25\% of the total contract value. The margin account pays 2\%, and the risk-free rate is 6\%. The index value today is \$3400. On the following days, you observe your investment.
	\begin{itemize}
	\item After 1 day, the index rose to \$3340
	\item After 2 days, the index dropped to \$3210
	\item After 3 days, the index rose to \$3410
	\end{itemize}
You decide to exit your position after 3 days. What is your payoff, profit, and rate of return?
\end{example}

\begin{solution}
We begin by determining the margin deposit: $250\cdot 3400\cdot 0.25=212,500$. Now we must compute the chages that occur from marking to market.
	\begin{align*}
	212,500e^{0.02\cdot 1/365}+250(3340-3400)&=197,511.64\\
	197,511.64e^{0.02\cdot 1/365}+250(3210-3340)&=165,022.47\\
	165,022.47e^{0.02\cdot 1/365}+250(3410-3210)&=215,031.51
	\end{align*}
The payoff is the last value listed in the computations, $\$215,031.51$. We arrive at the profit by taking the difference
	\begin{equation*}
	215,031.51-212,500e^{0.06\cdot 3/365}=2426.69.
	\end{equation*}
We can also compute the rate of return on the futures investment.
	\begin{align*}
	212,500e^{\alpha\cdot 3/365}&=215,031.51\\
	\alpha&=1.441
	\end{align*}
\end{solution}

When dealing with futures contract in such a short term, it is almost useless to discuss the rate of return of the investment. It will almost always be exaggerated. Time frames over a couple of weeks will probably be more meaningful. The result of the above should be clear, futures contracts can be a wild ride! We saw our investment almost evaporate after one day of poor performance in the market. The only thing this example didn't cover was a maintenance margin.

\begin{question}
Suppose you have the same futures arrangement as in the previous example. In addition, there is a maintenance margin of 80\% of the initial margin value. Determine the payoff and profit under this scenario.
	\begin{prompt}

	$\text{The payoff is }\answer{262511.64}$

	$\text{The profit is }\answer{2421.49}$

	\end{prompt}

\end{question}

\begin{solution}
Many of the computations are similar. The difference comes after the second day. You must bring a $212,500-165,022.47=47,477.53$ in collateral to your account. The calculation after the third day would be
	\begin{equation*}
	212,500e^{0.02\cdot 1/365}+250(3410-3210)=262,511.64
	\end{equation*}
It follows that your payoff is $\$262,511.64$. Your profit calculation is a bit messier.
	\begin{equation*}
	262,511.64-47,477.53e^{0.06\cdot 1/365}-212,500e^{0.06\cdot 3/365}=2421.49
	\end{equation*}
\end{solution}

As it should be expected, the margin call will result in reduced profits.


 
\end{document}


