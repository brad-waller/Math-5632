\documentclass{ximera}

\title{The Black-Scholes Model}
\author{Brad Waller}

\begin{document}
\begin{abstract}
\end{abstract}
\maketitle

Our last chapter dealt with the binomial models. These were very useful in developing prices for almost every derivative we have discussed thus far. The only drawback to the binomial model is that there is a choice that needs to be made: ``How many periods does my model need?" This kind of choice can be troubling and must often be defended. Fortunately, there is an alternative.

The Black-Scholes model is the continuous version of the Binomial model. In fact, the three named models we discussed in the previous chapter will converge to this particular model. In the early sections of this chapter, we will deal with the mechanics of the model in question, including a development of the Black-Scholes formula. The chapter will conclude with a variety of examples over several sections.

\end{document}
