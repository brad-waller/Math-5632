\documentclass{ximera}

%\documentclass{ximera}

\usepackage{float}
\usepackage{subcaption}

\pgfplotsset{compat=1.16}

\newtheorem{ass}{Assumption}

\def\check{\tikz\fill[scale=0.4](0,.35) -- (.25,0) -- (1,.7) -- (.25,.15) -- cycle;}





\outcome{Learn about the terms ``long'' and ``short.''}

\author{Brad Waller}

%Section 2.2

\title{The Long and the Short}

\begin{document}

\begin{abstract}
Calls and puts are financial assets on their own. Since they are derivatives, their price is determined by their underlying asset. Terms regarding this relationship are given in this section.
\end{abstract}

\maketitle

We have seen the term ``Short Sale'' in the past, but we did not go into the etymology at all. With the study of derivatives and, more specifically, options, it is good to know how your derivatives will react to changes in the price of the underlying asset. 

\begin{definition}
You are said to be {\bf long} in an asset if you benefit by its increase in value, and you are said to be {\bf short} in an asset if you benefit by its decrease in value.
\end{definition}

The short sale uses this terminology. Every short sale has an underlying asset. When you are the short seller, you are short in the underlying asset. 

Things are a little more interesting/convoluted when you are dealing with derivatives. This is because derivatives are also assets! We have only really dealt with three derivatives: forward contracts (futures), calls, and puts. Let's discuss each with respect to their (respective) underlying assets.

\begin{center}
\begin{tabular}{cccc}

& Forward Contract & Call Option & Put Option\\
\hline
	&			&			&		\\
Buyer & Long in 		& Long in 		& Short in 	\\
	& underlying asset	& underlying asset 	& underlying asset\\
	&			&			&		\\
Seller	& Short in 		& Short in 		& Long in\\
	& underlying asset 	& underlying asset 	& underlying asset
\end{tabular}
\end{center}

In addition to these positions, the buyer is always long in the derivative while the seller is short. There is not much to ask with regard to these concepts, but they do creep up in the launguage of our subject with some regularity. For example, it might be said that you enter into a long position on a call. Thet just means you are buying a call. However, I will never say something like you are short in a put. I feel that there is ambiguity in this, and it should be avoided. I would prefer to say you write a put.

\end{document}

