\documentclass{ximera}

%\documentclass{ximera}

\usepackage{float}
\usepackage{subcaption}

\pgfplotsset{compat=1.16}

\newtheorem{ass}{Assumption}

\def\check{\tikz\fill[scale=0.4](0,.35) -- (.25,0) -- (1,.7) -- (.25,.15) -- cycle;}





%\colorlet{regionColor}{black!30!white}

\outcome{We will discover how to construct the Capital Asset Pricing Model!}

\author{Brad Waller}

%Section 6.3

\title{The Capital Asset Pricing Model}

\begin{document}

\begin{abstract}
Using our market portfolio from the previous section, we will discover how expected rates of return are related to the market portfolio. 
\end{abstract}

\maketitle

Our last section dealt with the consequences of the inclusion of a risk-free rate in our portfolio theory. Investorw would all gravitate towards one portfolio: the market portfolio. There are still other portfolios in existance; the market portfolio might be hard to put together for the average investor. There is a relationship between expected returns of those portfolios and that of the market portfolio. The following image should help in constructing such a relationship.


\begin{center}
\begin{tikzpicture}
  \begin{axis}[
      xmin=-0.1,
      xmax=1,
      ymin=-0.1,
      ymax=.39,
      clip=true,
      %unit vector ratio*=1 1 1,
      axis lines=center,
      %grid = major,
      %ytick={-21,-18,...,21},
      %xtick={-21,-18,...,21},
      ticks=none,
      xlabel=$\sigma$,
      ylabel=${E[R]}$,
      y tick label style={anchor=west},
      every axis y label/.style={at=(current axis.above origin),
        anchor=south},
      every axis x label/.style={at=(current axis.right of origin),anchor=west},
      axis on top,
    ]
    \foreach \u in {.55,.6,...,2}{
      \addplot[line width=1pt,
        regionColor,
        variable=\t,domain=-3:2,smooth] (
            {
              sqrt(.36-.96*\u-.6*\t+.9*\u*\t+.69*\u^2+.4*\t^2)
            },
            {
              .18-.14*\u-.08*\t
            });
            
    }

   \foreach \u in {-.1,-.05,...,1}{
      \addplot[line width=1pt,
        regionColor,
        variable=\t,domain=-2:3,smooth] (
            {
              sqrt(.36-.96*\u-.6*\t+.9*\u*\t+.69*\u^2+.4*\t^2)
            },
            {
              .18-.14*\u-.08*\t
            });
            
   }
    
    \addplot[draw=regionColor,fill=regionColor,fill opacity=1,domain=-1:1,mark=none,smooth] ({sqrt(1+140*x^2)-.84},{x+.08}) \closedcycle;
   
    \addplot[line width=1pt,
        black,
        variable=\t,domain=-2:3,smooth] (
            {
              sqrt(.36-.96*0-.6*\t+.9*0*\t+.69*0^2+.4*\t^2)
            },
            {
              .18-.14*0-.08*\t
            });
    \node[circle,fill,inner sep=2pt] at (axis cs:.4,.1) {};
    \node[below left] at (axis cs:.4,.1) {$X$};

    \node[circle,fill,inner sep=2pt] at (axis cs:.46,.155) {};
    \node[above] at (axis cs:.46,.155) {$M$};

    \addplot[domain=0:1] {((.155-.068)/.46)*(x-.46)+.157};

    \node at (axis cs:0,.068) {--};
    \node[left] at (axis cs:0,.068) {$R_f$};
  \end{axis}
\end{tikzpicture}
\end{center}

The tangent line in the diagram is both tangent to the boundary of the feasible region (without the introduction of the risk-free rate), and it is tangent to the curve consisting of portfolios that are mixtures of the market portfolio and some asset/portfolio denoted $X$. That means that we know the slope of the tangent line to the curve connecting $X$ to $M$. It is the Sharpe ratio of the market portfolio!

For our purposes, we will parameterize this curve using $\alpha$. An arbitrary portfolio, $Q$, on this curve will have weight $\alpha$ in the market portfolio and weight $1-\alpha$ in $X$. In other words:

\[
Q=\alpha M+(1-\alpha)X
\]

and

\[
R_q=\alpha R_M+(1-\alpha)R_X.
\]

Using this information, we know that points $Q$ on our curve must come from their expected return and variance.

\begin{align*}
\mathbb{E}[R_q] 		&=\alpha\mathbb{E}[R_M]+(1-\alpha)\mathbb{E}[R_X]\\
\text{Var}(R_q) 		&=\alpha^2\text{Var}(R_M)+(1-\alpha)^2\text{Var}(R_X)+2\alpha(1-\alpha)\text{Cov}(R_M, R_X)\\
\sigma_q 			&=\sqrt{\text{Var}(R_q)}
\end{align*}

The point $Q$ will be at $(\sigma_q, \mathbb{E}[R_q])$, which are both functions of $\alpha$. If we want to compute the slope of the tangent line at $M$ to the curve connecting $X$ to $M$, we must compute the following derivative:

\[
\frac{\mathrm{d}\mathbb{E}[R_q]}{\mathrm{d}\sigma_q}.
\]

Unfortunately, this is very difficult. Fortunately, we were able to parameterize this curve in terms of $\alpha$.

\begin{align*}
\frac{\mathrm{d}\mathbb{E}[R_q]}{\mathrm{d}\sigma_q} 	&=\frac{\mathrm{d}\mathbb{E}[R_q]/\mathrm{d}\alpha}{\mathrm{d}\sigma_q/\mathrm{d}\alpha}\\
									&=\frac{\mathbb{E}[R_M]-\mathbb{E}[R_X]}{[2\alpha\text{Var}(R_M)-2(1-\alpha)\text{Var}(R_X)+(2-4\alpha)\text{Cov}(R_M, R_X)]/2\sigma_q}\\
									&=\frac{(\mathbb{E}[R_M]-\mathbb{E}[R_X])\sigma_q}{\alpha\text{Var}(R_M)-(1-\alpha)\text{Var}(R_X)+(1-2\alpha)\text{Cov}(R_M, R_X)}
\end{align*}
 
We evaluate this first derivative when $\alpha=1$ because that corresponds to the point $M$ on the curve connecting $X$ to $M$. This yields the following:

\[
\frac{(\mathbb{E}[R_M]-\mathbb{E}[R_X])\sigma_M}{\text{Var}(R_M)-\text{Cov}(R_M, R_X)}.
\]

$\sigma_q$ became $\sigma_M$ upon making the substitution $\alpha=1$. As we observed earlier, we actually know the value of this first derivative. It is the Sharpe ratio of the market portfolio:

\[
\phi_M=\frac{\mathbb{E}[R_M]-R_f}{\sigma_M}.
\]

The Capital Asset Pricing Model follows by equating these two formulae. 

\begin{align*}
\frac{(\mathbb{E}[R_M]-\mathbb{E}[R_X])\sigma_M}{\text{Var}(R_M)-\text{Cov}(R_M, R_X)} 	&=\frac{\mathbb{E}[R_M]-R_f}{\sigma_M}\\
\mathbb{E}[R_M]-\mathbb{E}[R_X] 										&=(\mathbb{E}[R_M]-R_f)\left[\frac{\text{Var}(R_M)-\text{Cov}(R_M, R_X)}{\sigma_M^2}\right]\\
\mathbb{E}[R_M]-\mathbb{E}[R_X] 										&=(\mathbb{E}[R_M]-R_f)(1-\text{Cov}(R_M, R_X)/\sigma_M^2)
\end{align*}

We will make the assignment 

\[
\beta_X=\frac{\text{Cov}(R_M, R_X)}{\sigma_M^2}.
\]

Now we may continue our argument.

\begin{align*}
\mathbb{E}[R_M]-\mathbb{E}[R_X] 										&=(\mathbb{E}[R_M]-R_f)(1-\beta_X)\\
\mathbb{E}[R_M]-\mathbb{E}[R_X] 										&=\mathbb{E}[R_M]-\mathbb{E}[R_M]\beta_X-R_f+R_f\beta_X\\
-\mathbb{E}[R_X] 													&=-\mathbb{E}[R_M]\beta_X-R_f+R_f\beta_X\\
\mathbb{E}[R_X] 													&=R_f+\beta_X(\mathbb{E}[R_M]-R_f)
\end{align*}

\begin{theorem}[The Capital Asset Pricing Model]
Let $X$ be a risky asset (or portfolio) and $M$ represent the market portfolio. Then we can relate the expected returns for each through the equation
\[
\mathbb{E}[R_X]=R_f+\beta_X(\mathbb{E}[R_M]-R_f)
\]
where 
\[
\beta_X=\frac{\text{Cov}(R_M, R_X)}{\sigma_M^2}.
\]
\end{theorem}

Since $\beta$ has numerator based on covariance and a denominator that is constant, we can infer a nice property of $\beta$. Let $P$ be a portfolio consisting of $X_1$, $X_2$, ... and $X_n$, with weights $\alpha_1$, $\alpha_2$, ... and $\alpha_n$, respectively. Then we have

\[
\beta_p=\sum_{j=1}^n \alpha_j\beta_j.
\]

Let's use this information.

\begin{question}
Suppose that you have two investments, $X_1$ and $X_2$ with weights $\alpha_1=2/3$ and $\alpha_2=1/3$. You know that $\beta_1=2.1$, $\beta_2=-0.6$, $\mathbb{E}[R_M]=0.09$ and $R_f=0.02$. What is the expected return of your portfolio?
\[
\mathbb{E}[R_p]=\answer{0.104}
\]
\end{question}
\begin{solution}
We must start by computing the $\beta$ value of the portfolio.
	\begin{align*}
	\beta_p 	&=\frac{2}{3}\beta_1+\frac{1}{3}\beta_2\\
			&=1.2
	\end{align*}

Now we apply the Capital Asset Pricing Model.
	\begin{align*}
	\mathbb{E}[R_X] 	&=R_f+\beta_p(\mathbb{E}[R_M]-R_f)\\
				&=0.02+1.2(0.09-0.02)\\
				&=0.104
	\end{align*}
	
\end{solution}

One lingering question may be: ``Why would I invest in an asset with a negative $\beta$?'' The answer has to do with hedging. Sometimes the market will perform poorly. An asset that has a return negatively correlated with the market will often perform well during these times. That is why the question had such an asset!
\end{document}

































