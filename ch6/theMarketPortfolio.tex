\documentclass{ximera}

%\documentclass{ximera}

\usepackage{float}
\usepackage{subcaption}

\pgfplotsset{compat=1.16}

\newtheorem{ass}{Assumption}

\def\check{\tikz\fill[scale=0.4](0,.35) -- (.25,0) -- (1,.7) -- (.25,.15) -- cycle;}





%\colorlet{regionColor}{black!30!white}

\outcome{Study the consequences of the introduction of a risk-free rate.}

\author{Brad Waller}

%Section 6.2

\title{The Market Portfolio}

\begin{document}

\begin{abstract}
This section deals with risk-free rates. Some interesting consequences are that the efficient portfolios consist only of two assets: the risk-free investment and something called the ``market portfolio."
\end{abstract}

\maketitle

Suppose that $P$ is some portfolio in our feasible region. Let $R_f$ denote our risk-free rate. We want to see what happens to portfolios that are mixtures of $P$ and the risk-free asset. $Q$ will be the portfolio that has weight $\alpha$ in $P$ and $1-\alpha$ in $R_f$. 

\begin{align*}
\text{Var}(R_q) 		&=\text{Var}(\alpha R_p+(1-\alpha)R_f)\\
				&=\alpha^2\text{Var}(R_p)+(1-\alpha)^2\text{Var}(R_f)+2\alpha(1-\alpha)\text{Cov}(R_p, R_f)\\
				&=\alpha^2\text{Var}(R_p)+2\alpha(1-\alpha)\text{Cov}(R_p, R_f)\\
				&=\alpha^2\text{Var}(R_p)+2\alpha(1-\alpha)(\mathbb{E}[R_pR_f]-\mathbb{E}[R_p]\mathbb{E}[R_f])\\
				&=\alpha^2\text{Var}(R_p)+2\alpha(1-\alpha)(\mathbb{E}[R_p]R_f-\mathbb{E}[R_p]R_f)\\
				&=\alpha^2\text{Var}(R_p)\\
\sigma_q 			&=|\alpha|\sigma_p
\end{align*}

This means that the standard deviation of the return of portfolio $Q$ only depends on the magnitude of the weight in $P$ and the standard deviation of the return of $P$. That's some really useful information. In fact, once of the consequences of this fact is that the curve consisting of such points $Q$ is a line, or more precisely the curve is two rays. One curve is a ray that connects the points $R_f$ and $Q$; it has slope

\[
\frac{\mathbb{E}[R_q]-R_f}{\sigma_q}.
\]

The other curve is a ray that starts at $R_f$ and has slope 

\[
-\frac{\mathbb{E}[R_q]-R_f}{\sigma_q}.
\]

This is illustrated in the cartoon below. 

\begin{center}
\begin{tikzpicture}

%this bit here fills in the region
\begin{scope}
     \fill[use Hobby shortcut,pattern=north east lines,pattern color=gray!50]
     (1,2) .. (1.3,2.5) .. (2,3) ..(5,4) --
     (5,4) .. (4,2) .. (4,.7) --
     (4,.7) .. (2,1.25) .. (1,2);
\end{scope}

%this bit here is the axes
\draw [-latex,
    thick] 
    (0,0) -- (6,0) node[right] {$\sigma$};
\draw [-latex,
    thick]
    (0,0) -- (0,5) node[above] {$E[R]$};

%these bits are the boundary; tikz will draw bezier curves through the given points
\draw[thick,
    red,
    use Hobby shortcut]
    (1,2) .. (1.3,2.5) .. (2,3) .. (5,4);

\path[draw,
    use Hobby shortcut,
    thick,
    blue]
    (1,2) .. (2,1.25) .. (4,.7);

\path [draw,
    use Hobby shortcut,
    thick,
    white]
    (4,.7) .. (4,2) .. (5,4);
    
%these bits are the dots/labels    
%\draw [fill=black] (5,4) circle (1pt) node[anchor=south west] {$A$};
%\draw [fill=black] (4,.7) circle (1pt) node[anchor=north west] {$B$};
%\draw [fill=black] (1,2) circle (1pt) node[anchor=south east] {$C$};
%\draw [fill=black] (1,2) circle (1pt) node[above=3pt,fill=white,inner sep=0pt] {$C$};
    \draw [fill=black] (3,3) circle (1pt) node[anchor=north west] {$Q$};
    \draw [fill=black] (0,2.5) circle (1pt) node[anchor=east] {$R_f$};
    \draw[->, >=latex] (0,2.5) -- (6,3.5);
    \draw[->, >=latex] (0,2.5) -- (6,1.5);
%\node [fill=white,inner sep=0pt] at (2.7,2.3) {$R$};

%for help drawing the curves, I initially put a grid over top the pic
%\draw (0,0) grid (6,5);
\end{tikzpicture}

\end{center}

The slope of the ray that passes through $R_f$ and $Q$ has a special name. It is called the Sharpe ratio.

\begin{definition}
The {\bf Sharpe ratio} is the slope of the line connecting the risk-free rate with the position of the portfolio. Explicitly, this is the quantity
\[
\phi_p=\frac{\mathbb{E}[R_p]-R_f}{\sigma_p}.
\]
\end{definition}

It should be clear that higher slopes are beneficial. This is because you are getting a better return for each unit of volatility that you are taking on. It will be our goal to maximize the Sharpe ratio. In general, this is quite an undertaking. If we are in the situation where there are only two risky assets, then we can parameterize all possible portfolios using one variable. This makes the maximization problem much simpler.

In this situation, we will be creating a tangent line starting at the risk-free rate and bouncing off the feasible region. Let's try an example.

\begin{example}
Suppose that we are given two risky assets, $X_1$ and $X_2$. We are given that $\text{Var}(R_1)=0.06$, $\text{Var}(R_2)=0.24$, $\text{Cov}(R_1, R_2)=-0.06$, $\mathbb{E}[R_1]=0.04$, $\mathbb{E}[R_2]=0.2$, and $R_f=0.024$. Maximize the Sharpe Ratio.
\end{example}

\begin{solution}
By definition, the sharpe ratio is determined from the risk-free rate and a portfolio. Fortunately, this example is related to one from the previous section, so we don't need to recompute everything.

\begin{align*}
\phi_p 	&=\frac{0.2-0.16\alpha-0.024}{\sqrt{0.42\alpha^2-0.6\alpha+0.24}}\\
		&=\frac{0.176-0.16\alpha}{\sigma_p}
\end{align*}

This slope is a function of $\alpha$, so we can determine its maximum by using first derivatives.

\begin{align*}
\frac{\mathrm{d}\phi_p}{\mathrm{d}\alpha} 	&=\frac{-0.16\sigma_p-(0.176-0.16\alpha)(0.84\alpha-0.6)/(2\sigma_p)}{\sigma_p^2}\\
							&=\frac{-0.32\sigma_p^2-(0.176-0.16\alpha)(0.84\alpha-0.6)}{2\sigma_p^3}\\
							&=\frac{-0.32(0.42\alpha^2-0.6\alpha+0.24)-(0.176-0.16\alpha)(0.84\alpha-0.6)}{2\sigma_p^3}\\
							&=\frac{-0.05184\alpha+0.0288}{2\sigma_p^3}
\end{align*}

When we set this equation equal to $0$, we get that there is a critical point corresponding to $\alpha=5/9$. This critical point is a maximum, so we have succeeded!
\end{solution}

What have we succeeded in doing here? We showed that there is precisely one portfolio consisting of risky assets that is efficient. This has a special name; it is called the {\bf market portfolio}. It has this name since all investors will gravitate to portfolios that consist of the market portfolio and the risk-free asset, depending on risk-appetite. The introduction of the risk-free rate has greatly expanded our feasible region. The cartoon below illustrates this expansion.

\begin{center}
\scalebox{.8}{
\begin{tikzpicture}

%this bit here fills in the region
\begin{scope}
     \fill[pattern=north east lines,pattern color=gray!50]
	(0,1.8) -- (5,4.875)--
	(5,4.875) -- (5,-1.275)--
	(5,-1.275) --(0,1.8);
\end{scope}

%this bit here is the axes
\draw [-latex,
    thick] 
    (0,0) -- (6,0) node[right] {$\sigma$};
\draw [-latex,
    thick]
    (0,0) -- (0,5) node[above] {$E[R]$};

%these bits are the boundary; tikz will draw bezier curves through the given points
\draw[thick,
    use Hobby shortcut]
    (1,2) .. (1.3,2.5) .. (2,3) .. (5,4);

\path[draw,
    use Hobby shortcut,
    thick]
    (1,2) .. (2,1.25) .. (5,.7);

\path [draw,
    use Hobby shortcut,
    thick,
    white]
    (5,.7) .. (5,2) .. (5,4);

\draw [-latex, red, thick] (0,1.8) -- (5,4.875);

%\draw [-latex, thick] (0,1.8) -- (5,.2);
\draw [-latex, thick, blue] (0,1.8)--(5, -1.275);
%\draw [-latex, thick] (0,1.8)--(5,3.4);
    
%these bits are the dots/labels    
%\draw [fill=black] (5,4) circle (1pt) node[anchor=south west] {$A$};
%\draw [fill=black] (4,.7) circle (1pt) node[anchor=north west] {$B$};
\draw [fill=red] (1.74,2.86) circle (1pt) node[anchor=south east] {$M$};
\draw [fill=red] (0,1.8) circle (1pt) node[anchor=east]{$R_f$};
%\draw [fill=black] (2.77, .90) circle (1pt) node[anchor=north east]{$N$};
%\draw [fill=black] (1,2) circle (1pt) node[above=3pt,fill=white,inner sep=0pt] {$C$};

\node [fill=white,inner sep=0pt] at (2.7,2.3) {$R$};

%for help drawing the curves, I initially put a grid over top the pic
%\draw (0,0) grid (6,5);
\end{tikzpicture}
}
\end{center}

 
\end{document}

































